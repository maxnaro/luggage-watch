\documentclass{report}

\usepackage[a4paper, margin=3cm]{geometry}
\usepackage{multicol}
\usepackage{fancyhdr}
\usepackage{hyperref}
\usepackage{graphicx}
\usepackage{pdfpages}
\usepackage{lscape}

\fancyhf{}
\fancyhead[L]{Embedded System for Suspicious Luggage Detection}
\fancyhead[R]{Max Narongchai --- F226732}
\fancyfoot[C]{\thepage}
\pagestyle{fancy}

\begin{document}

\begin{multicols*}{2}

	{
		\centering
		{\LARGE{\bfseries January Deliverable Report}\\[6pt]}
		\vspace{.5cm}

		\begin{tabular}{c}
			\Large Max Narongchai \\ Computer Science \\ COC251 \\ F226732
		\end{tabular}
		\vspace{.5cm}

		{\small \today\par}
	}

	\section*{Progress}

	\subsection*{Report Writing}

	The current progress made on the report includes the completion of the
	introduction and literature review sections (which can be found in the attached
	section~\ref{appendix:report}).

	Subsequent sections, including requirements analysis have been outlined as
	future work to be completed in the following weeks.

	\subsection*{Dataset Procurement}

	Several datasets have been identified and procured for use in training and
	validation. I have been unable to procure the PETS and i-LIDS datasets due to
	the difficulties in sourcing them, but alternative datasets such as ABODA have
	been acquired.

	For training purposes, the COCO dataset has proved useful in providing large
	volumes of labelled images (luggage, people, bags, etc.) for initial model
	training.

	\subsection*{Programming \& Implementation}

	The development environment has been set up, with baselines YOLO models being
	trained on the COCO dataset. Initial experiments have led to promising results
	in mAP@50 and @95, indicating that the dataset and model choice are appropriate
	for the task. Further testing and fine-tuning will be conducted in the coming
	weeks.

	Future implementation plans include object tracking and anomaly detection
	algorithms such as Deep SORT, as well as integration with the embedded hardware
	platform. Moreover, communication with KTP partners to align on model
	compilation and optimisation with TensorRT is planned, along with a potential
	of implementing NVIDIA's DeepStream SDK for streamlined video analytics.

	\section*{Challenges}

	\subsection*{Dataset Availability}

	Procuring high-quality datasets specific to suspicious luggage detection has
	been difficult as the literature cites few publicly available datasets,
	especially those that are easily accessible in 2026. I have gotten in contact
	with multiple authors of relevant papers to request access to their datasets,
	but have yet to receive any responses. As a workaround, I have resorted to
	using more general datasets (e.g., COCO) for initial model training and
	validation, while continuing to search for more specific datasets for the
	future object tracking implementation.

	\subsection*{Obligations}

	Balancing project work with other academic and personal obligations has been
	challenging, especially on the lead-up to coursework deadline and examinations.
	To mitigate this, I have created a revised work plan (see
	figure~\ref{fig:gantt-chart}) that allocates more time for project work in the
	weeks following major deadlines, to ensure steady progress.

	\subsection*{Delays}

	Some delays have been encountered in the writing of the literature review
	section of the report. My unfamiliarity with the field meant that I had to
	spend additional time researching and understanding the relevant concepts and
	literature before being able to write effectively about them. I believe that
	the time invested in the literature review will pay off in the long run, as it
	has provided a solid foundation for the rest of the report and project.

	\section*{Changes in Direction}

	The initial pivot of the project's scope, from an embedded system for people
	counting and detecting to suspicious luggage detection, and the broadening of
	the scenario from train stations to general public places have been the two
	major changes in direction so far. Neither of these changes have affected the
	core objectives of the project or the intended implementations significantly,
	aside from requiring more effort in dataset procurement.

\end{multicols*}

\begin{landscape}
	\section*{Work Plan}

	\begin{figure}[!htb]
		\centering
		\includegraphics[height=0.8\textheight,width=\textwidth,keepaspectratio]{figures/plantuml/gantt.png}
		\caption{Work Plan as of January 2026}
		\label{fig:gantt-chart}
	\end{figure}
\end{landscape}

\appendix
\renewcommand{\thesection}{\Alph{section}}
\pagestyle{empty}
\includepdf[pages=1,pagecommand={\section{Report}\label{appendix:report}}]{report.pdf}
\includepdf[pages=2-,pagecommand={}]{report.pdf}

\end{document}