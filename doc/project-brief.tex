\documentclass{article}
\usepackage{fancyhdr}
\usepackage{hyperref}

\pagestyle{fancy}
\fancyhf{}
\fancyfoot[L]{F226732}
\fancyfoot[R]{Page \thepage}

\title{Project Brief: Embedded System for Suspicious Luggage Detection}
\author{Max Narongchai}
\date{October 24, 2025}

\begin{document}

\maketitle
\thispagestyle{empty}
\newpage

\tableofcontents
\newpage

\section{Introduction}

This project aims to develop an embedded system that can detect suspicious, or unattended, luggage from video surveillance footage.
The system will utilise computer vision and deep learning techniques to identify luggage distinctly from other objects in the scene.

Primarily, the use case is focused on train carriages, where unattended luggage can pose inconvenience for passengers and, at worst, security risks. 

\section{Objectives}

The main objectives of this project are:
\begin{itemize}
    \item Develop deep learning software capable of detecting, and recognising, objects and people from a camera footage in real time.
    \subitem Aiming for ~85\% precision in detecting suspicious luggage with a ~15\% rate of false positives; 2-5 second inference time, compromising with the limitations of hardware and the volume of passengers, and luggage, that may fill a train carriage at a given time
    \item Integrate the software within an embedded system, Jetson Orin Nano.
    \subitem This will include hardware integration, with camera inputs, \textit{proper power management, and networking for logging}. As well as, resource usage optimisation, managing system and GPU memory efficiently, and minimising CPU overhead. 
\end{itemize}

\section{Scope}

% Point of contention here, may need to reduce the scope of the project if datasets are not available
As mentioned in the introduction, the main use case of this project is in train carriages. 
This will include luggage left on the overhead storage compartments and luggage compartments (often situated near the entrance of each carriage). 
The software will, therefore, be optimised for precision in this environment; the aforementioned ~85\% precision applying.

\section{Technical Approach}

The project will be implemented on an NVIDIA Jetson Orin Nano embedded system, with the software developed in Python.
A majority of similar projects implement a version of a YOLO (You Only Look Once) model for object detection, which I believe will be the direction I take this project.
However, despite being a popular choice for object detection, I will evaluate other architectures such as SSD (Single Shot MultiBox Detector) and Faster R-CNN to determine the most suitable model for this application.

A dataset will be required to train the model, which will be collected from pre-existing, publicly available sources, and potentially supplemented by using proprietary data.
The format of the dataset is important in the model's application; training a model on conventional, RGB, data means that the object recognition will only work with data of the same format.
RGB, RGB-D, ToF (time-of-flight), all offer different information about the scene. The most likely format for this project is RGB due to the availability of datasets and ease of implementation.

\section{Risks \& Constraints}

The main risks in this project are my lack of experience in deep learning and computer vision, and the availability of a suitable dataset.

\section{Timeline}

I've decided to keep track of my progress using a \href{https://macksmacks.atlassian.net/jira/software/projects/FYP/boards/1}{Jira board}, which will help me manage tasks and deadlines effectively.
Using Jira provides a lot of different tools for project management, including a kanban board, and timeline which I will use to visualise the upcoming and current demands of the project.

\end{document}