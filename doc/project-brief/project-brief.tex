\documentclass{article}
\usepackage{fancyhdr}
\usepackage{graphicx}
\usepackage[hidelinks]{hyperref}
\usepackage{apacite}

\pagestyle{fancy}
\fancyhf{}
\fancyfoot[L]{F226732}
\fancyfoot[R]{Page \thepage}

\title{Project Brief: Embedded System for Suspicious Luggage Detection}
\author{Max Narongchai}
\date{October 24, 2025}

\begin{document}

\maketitle
\thispagestyle{empty}
\newpage

\tableofcontents
\newpage

\section{Introduction}

This project aims to develop an embedded system that can detect suspicious, or unattended, luggage from video surveillance footage.
The system will utilise computer vision and deep learning techniques to identify luggage distinctly from other objects in the scene.

Primarily, the use case is focused on public or semi-public spaces, such as transport hubs, waiting areas, or building lobbies. In these environments, unattended luggage can pose an inconvenience and, at worst, a security risk.

\section{Objectives}

The main objectives of this project are:
\begin{itemize}
    \item Develop deep learning software capable of detecting, and recognising, objects and people from a camera footage in real time.
    \begin{itemize}
        \item Evaluate and optimise the model to achieve a suitable balance between precision, recall, and false positive rates for a real-world surveillance application.
        \item Target a near-real-time inference speed, balancing detection accuracy with the computational constraints of the embedded hardware.
    \end{itemize}
    \item Integrate the software within an embedded system, Jetson Orin Nano.
    \begin{itemize}
        \item This will include hardware integration, with camera inputs, \textit{proper power management, and networking for logging}, as well as resource usage optimisation, managing system and GPU memory efficiently, and minimising CPU overhead.
    \end{itemize}
\end{itemize}

\section{Scope}

% Point of contention here, may need to reduce the scope of the project if datasets are not available
As mentioned in the introduction, the project's use case is in monitoring public or semi-public environments.
This will include detecting luggage (such as bags, backpacks, or suitcases) left unattended in open areas, near seating, or in other designated zones for a pre-defined period.
The software will, therefore, be developed as a more general-purpose solution for unattended object detection. Performance will be evaluated based on metrics suitable for this task, rather than a fixed precision target, and benchmarked against available public datasets.

\section{Literature Review}

I begin my review by exploring existing research and reviews in the field of unattended, or suspicious, object detection using computer vision and deep learning techniques.
This includes a broader view on machine learning techniques for suspicious object detection \cite{dubey2024critical}, as well as a specific, novel approach to abandoned object detection using deep learning methods \cite{qasim2024abandoned}.
The main question of the review is to determining how the specific implementation choices tackle the general challenges posed.

\subsection{Reading \& Summarisation}

\subsubsection[Abandoned Object Detection and Classification Using Deep Embedded Vision]
{``Abandoned Object Detection and Classification Using Deep Embedded Vision'' \protect\cite{qasim2024abandoned}}



\subsubsection[A Critical Study on Suspicious Object Detection with Images and Videos Using Machine Learning Techniques]
{``A Critical Study on Suspicious Object Detection with Images and Videos Using Machine Learning Techniques'' \protect\cite{dubey2024critical}}



\section{Approach}

The project will be implemented on an NVIDIA Jetson Orin Nano embedded system, with the software developed in Python.
A majority of similar projects \cite{qasim2024abandoned} implement a version of a YOLO (You Only Look Once) model for object detection, which I believe will be the direction I take this project.
However, despite being a popular choice for object detection, I will evaluate other architectures such as SSD (Single Shot MultiBox Detector) and Faster R-CNN to determine the most suitable model for this application.

A dataset will be required to train the model, which will be collected from pre-existing, publicly available sources, and potentially supplemented by using proprietary data.
The format of the dataset is important in the model's application; training a model on conventional, RGB, data means that the object recognition will only work with data of the same format.
RGB, RGB-D, ToF (time-of-flight), all offer different information about the scene. The most likely format for this project is RGB due to the availability of datasets and ease of implementation.

\section{Risks \& Constraints}

The main risks in this project are my lack of experience in deep learning and computer vision, and the availability of a suitable, high-quality labelled dataset for unattended object detection.

\section{Timeline}

I've decided to keep track of my progress using a \href{https://macksmacks.atlassian.net/jira/software/projects/FYP/boards/1}{Jira board}, which will help me manage tasks and deadlines effectively.
Using Jira provides a lot of different tools for project management, including a kanban board, and timeline which I will use to visualise the upcoming and current demands of the project as shown in Figure 1.

\begin{figure}[ht]
\centering
\includegraphics[width=\linewidth]{../figures/luggage_watch_2025-10-20_09.47pm.png}
\caption{Initial project timeline.}
\end{figure}

\bibliographystyle{apacite}
\bibliography{project-brief}

\end{document}