\documentclass{report}

\usepackage{fancyhdr}
\usepackage{natbib}
\usepackage[a4paper, margin=3cm]{geometry}
\usepackage{graphicx}
\usepackage{csquotes}
\usepackage{multicol}

\fancyhf{}
\fancyhead[L]{Embedded System for Suspicious Luggage Detection}
\fancyhead[R]{Max Narongchai --- F226732}
\fancyfoot[C]{\thepage}
\pagestyle{fancy}

\bibliographystyle{agsm}

\begin{document}

\begin{titlepage}
	\centering
	\vspace*{2.5cm}

	{\LARGE{Embedded System for}\\[6pt]
		\Huge\bfseries Suspicious Luggage Detection\par}

	\vspace{1.5cm}

	\begin{tabular}{c}
		\Large Max Narongchai \\ Computer Science \\ COC251 \\ F226732
	\end{tabular}

	\vfill

	{\small \today\par}
\end{titlepage}

\begin{abstract}
	% TODO: complete at the end

	\noindent\textbf{Keywords:} suspicious luggage, CCTV, embedded system, anomaly detection
\end{abstract}

\tableofcontents
\newpage

\begin{multicols*}{2}
	\section{Introduction}

	Surveillance is a powerful tool, used in various public settings to ensure
	safety of the public on a daily basis. CCTV, closed circuit television, is
	employed widely with operators having to comb through hours of footage everyday
	to uphold security. This is a tedious task, and human error is inevitable.
	Posing the question, can we automate this process?

	One of the many threats to public safety is unattended luggage, this is a
	common occurence which, at best, causes inconvenience, and at worst, could be
	placed for a more sinister purpose.

	An automated approach to detecting suspicious luggage could alleviate the
	burden on operators, allowing them to dedicate more attention to pressing
	matters; require smaller teams to monitor larger areas, saving resources; and
	reduce human error, bolstering overall security.

	\section{Literature Review}

	\subsection{Object Detection}

	Object detection is a computer vision task that involves identifying and
	locating objects within an image or video frame.

	\subsection{Existing Systems}

	This section explores existing research and reviews in the field of unattended,
	or suspicious, object detection using computer vision and deep learning
	techniques. This includes a broader view on machine learning techniques for
	suspicious object detection \citep{dubey2024critical}, as well as a specific,
	novel approach to abandoned object detection using deep learning methods
	\citep{qasim2024abandoned}. The main question of the review is to determining
	how the specific implementation choices tackle the general challenges posed.

	\subsubsection[A Critical Study on Suspicious Object Detection with Images and Videos Using Machine Learning Techniques]
	{\enquote{A Critical Study on Suspicious Object Detection with Images and Videos Using Machine Learning Techniques} \protect\citep{dubey2024critical}}

	Authors \citeauthor{dubey2024critical} provide a broad study of the field,
	reviewing machine learning techniques for detecting suspicious objects. The
	authors cover a range of detection methods, from static images, real-time
	videos, and via IoT systems.

	A pertinent point raised in this review are the challenges that researchers
	faced in this domain \enquote{illumination changes, occlusion, noise, poor
		resolution, and real-time processing complexities}.

	The review compares multiple deep learning models with the most promising
	results coming from Faster-RCNN, Mask-RCNN, and variants of YOLO. Despite the
	promise of these models, the authors conclude that many limitations are still
	faced, particularly in terms of accuracy in busy scenes.

	\subsubsection[Abandoned Object Detection and Classification Using Deep Embedded Vision]
	{\enquote{Abandoned Object Detection and Classification Using Deep Embedded Vision} \protect\citep{qasim2024abandoned}}

	The article by \citeauthor{qasim2024abandoned} present a novel approach for
	identifying abandoned objects in surveillance footage. It features a two-stage
	approach, the first stage employing a ConvLSTM (Convolutional Long Short-Term
	Memory) model which combines the abilities of a CNN and LSTM to capture both
	spatial and temporal features from video data, classifying scenes as
	\enquote{suspicious} or \enquote{non-suspicious}; the second stage, given a
	suspicious scene, utilises a YOLOv8l model to classify the detected objects.

	\subsubsection{Identifying the Research Gap}

	The two studies provide an answer to the review's main question. The review by
	\citeauthor{dubey2024critical} establishes the persistent challenges of the
	domain: environmental factors, and, most critically for this project, the
	difficulty of \enquote{real-time processing complexities}
	\citep{dubey2024critical}. The implementation choices of researchers are a
	response to these challenges.

	A state-of-the-art example is presented by \citeauthor{qasim2024abandoned}, who
	tackle these challenges with a novel, specific implementation: a two-stage
	model \citep{qasim2024abandoned}. Their ConvLSTM classifier acts as an
	efficient first-pass filter, analysing spatio-temporal data to identify
	\enquote{suspicious} scenes. This implementation choice addresses the challenge
	of real-time processing by preventing the system from running the more
	computationally expensive YOLOv8l model on every single frame. This solution
	allows the system to achieve high accuracy (99.70\% for the localiser) by
	focusing resources only when necessary.

	However, this is where the research gap, and the justification for this
	project, becomes clear. While the \citeauthor{qasim2024abandoned} paper is
	titled \enquote{Using Deep Embedded Vision}, their experimental setup was
	conducted on a high-performance NVIDIA GeForce RTX 3090 GPU. This hardware,
	while having been superseded, remains a high-power, high-cost component not
	suitable for a typical, low-cost embedded application, revealing a gap between
	academic demonstration and practical, real-world deployment.

	Therefore, this project is justified as a critical investigation into the
	practical implementation and optimisation of an object detection model. It will
	tackle the \enquote{real-time processing complexities} identified by
	\citeauthor{dubey2024critical} by moving from a theoretical, high-performance
	environment to a constrained one.

	\section{Requirements Analysis}

	\section{System Design}

	\section{Implementation}

	\section{Testing and Evaluation}

	\section{Conclusion}
\end{multicols*}

\bibliography{report}

\end{document}
