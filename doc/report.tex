\documentclass{report}

\usepackage{fancyhdr}
\usepackage{natbib}
\usepackage[a4paper, margin=3cm]{geometry}
\usepackage{graphicx}
\usepackage{csquotes}
\usepackage{multicol}
\usepackage{amsmath}
\usepackage{booktabs}
\usepackage{tabularx}
\usepackage{hyperref}

\fancyhf{}
\fancyhead[L]{Embedded System for Suspicious Luggage Detection}
\fancyhead[R]{Max Narongchai --- F226732}
\fancyfoot[C]{\thepage}
\pagestyle{fancy}

\bibliographystyle{agsm}

\renewcommand\tabularxcolumn[1]{m{#1}} % make X columns vertically centered

\begin{document}

\begin{titlepage}
	\centering
	\vspace*{2.5cm}

	{\LARGE{Embedded System for}\\[6pt]
		\Huge\bfseries Suspicious Luggage Detection\par}

	\vspace{1.5cm}

	\begin{tabular}{c}
		\Large Max Narongchai \\ Computer Science \\ COC251 \\ F226732
	\end{tabular}

	\vfill

	{\small \today\par}
\end{titlepage}

\begin{abstract}
	% TODO: complete at the end

	\noindent\textbf{Keywords:} suspicious luggage, CCTV, embedded system, anomaly detection
\end{abstract}

\tableofcontents
\newpage

\begin{multicols*}{2}
	\section{Introduction}

	Surveillance is a powerful tool, used in various public settings to ensure
	safety of the public on a daily basis. CCTV, closed circuit television, is
	employed widely with operators having to comb through hours of footage everyday
	to uphold security. This is a tedious task, and human error is inevitable.
	Posing the question, can we automate this process?

	One of the many threats to public safety is unattended luggage, this is a
	common occurence which, at best, causes inconvenience, and at worst, could be
	placed for a more sinister purpose. Numerous incidents at transportation hubs
	and public spaces have demonstrated the potential dangers of unattended luggage
	and bags, with counter terrorism authorities advising the public and security
	personnel to remain vigilant and report any suspicious
	items~\citep{protectuk2022unattended}. Historical acts of terrorism, such as
	the Boston Marathon bombing, have emphasised the importance of detecting
	suspicious items in crowded spaces, highlighting the need for reliable
	detection mechanisms in these scenarios.

	Compared to other threats, unattended luggage is distinct in its ability to be
	discrete. To the human eye, with low resolution CCTV footage, a security threat
	may be indistinguishable from legitimate items, presenting a unique challenge
	compared to more obvious security threats such as disorderly conduct. False
	positives can increase the burden on teams responsible for monitoring, being
	too relaxed could have dire consequences.

	Systems recording footage at all hours of the day generate an overwhelming
	amount of data, the sheer volume posing a significant challenge for operators
	to review on a regular basis. This implies a costly requirement for large teams
	of personnel manually monitoring the footage, diverting resources from other
	facets of security, increasing operational expenses, and potentially leading to
	fatigue-induced oversights.

	Automation in security is already a widely incorporated concept accross the
	public sector, with systems such as facial recognition and intrusion detection
	providing results that manual monitoring alone cannot achieve; at least without
	significant human-resource investment. An automated approach to detecting
	suspicious luggage could alleviate the burden on operators, allowing them to
	dedicate more attention to pressing matters; require smaller teams to monitor
	larger areas, saving resources; and reduce human error, bolstering overall
	security.

	Edge processing using embedded systems has become an increasingly viable, cost
	effective, and practical solution for surveillance applications. Versus a cloud
	solution, edge processing offers lower latency, by way of being situated
	physically closer to the data source; improved privacy, sensitive data remains
	on site and the onus is on the operator to secure it --- not a cloud provider;
	and network independence, there is no reliance on a stable internet connection
	just the local network~\citep{pamadi2025edgevscloud}. The proliferation of
	affordable, high-performance embedded hardware, with enhanced software support,
	has made a strong case for edge processing in surveillance; the benefits of
	cloud computing's scalability and power are minimised by new embedded
	solutions.

	Advances in computer vision technology and deep learning have made the issue of
	embedded suspicious luggage detection more tractable. Similar automation tasks,
	are being successfully implemented in real-world applications.

	Therefore, this project aims to design and implement an embedded system for
	suspicious luggage detection. The system will bridge the gap between academic
	research and practical deployment, focusing on real-time performance on edge
	devices. The target for the project is to create a deployable solution that can
	integrate with surveillance infrastructure, utilising an NVIDIA Jetson Orin
	Nano as a cost-effective piece of hardware suitable for the task.

	\section{Literature Review}

	\subsection{Core Concepts of Computer Vision and Object Detection}

	\subsubsection{The Evolution of Detection Techniques}

	Authors \citeauthor{zou2025comprehensive} provide a comprehensive review of
	object detection methods, describing a shift from traditional algorithms to
	modern deep learning and transformer-based approaches.

	The survey~\citep{zou2025comprehensive} displays that the progression of object
	detection techniques can be categorised into three main eras: traditional
	methods, deep learning-based methods, and state-of-the-art modern methods.

	\begin{itemize}
		\item \textbf{Traditional Methods:}
		      SIFT (scale-invariant feature transform),
		      HOG (histogram of oriented gradients), and Viola-Jones, are examples of
		      the earliest feature-based methods of object detection. These methods
		      paved the way for the field but struggled with limits of complexity,
		      computational resources, and adaptability to diverse environments.
		      Few models were suitable for real-time application, excluding HOG and Viola-Jones,
		      the latter of which introduced usage of Haar-like features which are still used
		      in some real-time applications due to their facilitation of rapid inference times.

		      In addition, decision tree classifiers (DTs), K-nearest neighbours (KNNs) and
		      support vector machines (SVMs) were developed to address the shortcomings of
		      feature-based methods using machine learning. The integration of ML with
		      traditional methods improved adaptability and accuracy, but left much to be
		      desired in terms of computational complexity and real-time performance.
		\item \textbf{Deep Learning-Based Methods:}
		      A step away from the \enquote{handcrafted}
		      feature sets of traditional methods, deep learning-based methods offered solutions
		      that automated feature extraction and learning.

		      Convolutional neural networks (CNNs) became the backbone of these methods,
		      leading to the development of two-stage detectors. Architectures such as R-CNN,
		      Fast R-CNN, and Faster R-CNN introduced region proposals combined with CNNs to
		      extract features (with Faster R-CNN going a step further by integrating region
		      proposal networks). This is to say that these models were a significant step
		      forward in terms of accuracy and adaptability yet remained computationally
		      intensive, limiting their real-time application.

		      The shortcomings of two-stage detectors led to the development of single-stage
		      detectors. SSDs and YOLO (You Only Look Once)
		      \citep{jocher2023ultralytics,pathak2023object} revolutionised the field by
		      removing the region proposal step, enabling predictions from an entire image in
		      a single propagation. This marked a breakthrough in real-time object detection,
		      and is used in nearly every example of abandoned luggage detection in
		      literature \citep{abandoned2025vrsalovic,dubey2024critical}.
		\item \textbf{Modern Methods:}
		      The latest advancements in object detection are represented by transformer-based
		      approaches. The application of transformers, initially for natural language
		      processing, then developing into vision transformers (ViTs), finally to detection
		      transformers (DETRs), has introduced a new pardigm in object detection.

		      Utilising self-attention mechanisms, mechanisms that take into account the
		      \enquote{importance} of features relative to one another in an
		      input~\citep{vaswani2023attentionneed}, DETRs predict objects without the need
		      for predefined anchor boxes (used in CNN-based, SSD, and some\footnote{YOLOv2,
			      v3, v4, v5, and v7~\citep{terven2023comprehensive}} YOLO models to propose
		      potential object locations). This allows for transformer-based models to
		      determine objects outside of the constraints of pre-set shapes and sizes,
		      improving adaptability to diverse object types and environments.

		      Other notable methods include hybrid models that combine CNNs and transformers,
		      multi-scale detection, and lightweight architectures such as
		      MobileNet~\citep{chiu2020mobilenet}.
	\end{itemize}

	The study demonstrates that the evolution of object detection has been driven
	by persistent challenges, the need to manage object occlusions, handling object
	scale, and improving real-time processing capabilities.

	\subsubsection{Evaluation Metrics for Object Detection}

	Progress through the eras of object detection has been measured using a
	consistent, standard set of metrics. The survey by
	\citeauthor{zou2025comprehensive} highlights the importance of these metrics in
	evaluating and includes the following key metrics found in
	table~\ref{tab:evaluation_metrics_desc} and
	table~\ref{tab:evaluation_metrics_formula}~\citep{zou2025comprehensive}.
\end{multicols*}

\begin{table}[ht]
	\centering
	\small
	\begin{tabularx}{\columnwidth}{@{\extracolsep{\fill}}>{\raggedright\arraybackslash}X >{\raggedright\arraybackslash}X >{\raggedright\arraybackslash}X}
		\toprule
		\textbf{Metric}                                                                                                                 & \textbf{Description}                                                                                        & \textbf{Insights}                                                                                                                                                                                          \\
		\midrule
		Intersection over Union ($\mathrm{IoU}$)                                                                                        & Measure of overlap between predicted and ground truth bounding boxes.                                       & Bounded by $[0,1]$, a higher IoU indicates better localisation (more accurate bounding box predictions).                                                                                                   \\
		True Positive ($\mathrm{TP}$) / False Positive ($\mathrm{FP}$) / True Negative ($\mathrm{TN}$) / False Negative ($\mathrm{FN}$) & Counts of correct and incorrect predictions.                                                                & Fundamental for calculating precision, recall, and F1-score.                                                                                                                                               \\
		Recall                                                                                                                          & The ratio of true positives to the sum of true positives and false negatives.                               & Bounded by $[0,1]$, higher recall indicates better detection of actual objects.                                                                                                                            \\
		Average Precision ($\mathrm{AP}$)                                                                                               & Area under the precision recall curve (plotted by evaluating the model at different confidence thresholds). & Bounded by $[0,1]$, a higher AP indicates better overall performance across all thresholds for a given class.                                                                                              \\
		Mean Average Precision ($\mathrm{mAP}$)                                                                                         & The mean of AP accross all classes.                                                                         & Bounded by $[0,1]$, a higher mAP indicates better overall performance across all classes.                                                                                                                  \\
		$\mathrm{F1}$-Score                                                                                                             & The harmonic mean of precision and recall.                                                                  & Bounded by $[0,1]$, a higher F1-score indicates a better balance between precision and recall. It is more useful when there is an uneven class distribution.                                               \\
		Matthews Correlation Coefficient ($\mathrm{MCC}$)                                                                               & A measure of classification quality, a numerical representation of a confusion matrix.                      & Bounded by [-1, 1], 1 indicating perfect prediction, 0 indicating random prediction, and -1 indicating total disagreement between prediction and observation. Particularly useful for imbalanced datasets. \\
		Cohen's Kappa ($\kappa$)                                                                                                        & A measure of accuracy that takes into account the probability of a random agreement.                        & Bounded by $[-1, 1]$, 1 indication perfect agreement, 0 equivalent to chance, and -1 indicating complete disagreement. It has the same use-cases as $\mathrm{MCC}$.                                        \\
		\bottomrule
	\end{tabularx}
	\caption{Evaluation metrics for object detection: descriptions and insights.}
	\label{tab:evaluation_metrics_desc}
\end{table}

\begin{table}[ht]
	\centering
	\small
	\begin{tabularx}{\columnwidth}{@{\extracolsep{\fill}}>{\raggedright\arraybackslash}X >{\raggedright\arraybackslash}X}
		\toprule
		\textbf{Metric}                                   & \textbf{Formula}                                                                                                                                                                                               \\
		\midrule
		Intersection over Union ($\mathrm{IoU}$)          & $$\mathrm{IoU} = \dfrac{|B_{\mathrm{pred}} \cap B_{\mathrm{gt}}|}{|B_{\mathrm{pred}} \cup B_{\mathrm{gt}}|}$$                                                                                                  \\
		Recall                                            & $$\mathrm{Recall} = \frac{\mathrm{TP}}{\mathrm{TP} + \mathrm{FN}}$$                                                                                                                                            \\
		Average Precision ($\mathrm{AP}$)                 & For threshold $n \in [0, 1]$, recall at threshold $R_n$, and precision at threshold $P_n$: $$\mathrm{AP} = \sum_n{(R_n - R_{n-1})\cdot P_n}$$                                                                  \\
		Mean Average Precision ($\mathrm{mAP}$)           & For number of classes $N$: $$\mathrm{mAP} = \frac{1}{N}\cdot\sum_{i=1}^N{\mathrm{AP}_i}$$                                                                                                                      \\
		$\mathrm{F1}$-Score                               & $$\mathrm{F1} = 2 \cdot \frac{P\cdot R}{P + R} = \frac{2\cdot \mathrm{TP}}{2\cdot \mathrm{TP} + \mathrm{FP} + \mathrm{FN}}$$                                                                                   \\
		Matthews Correlation Coefficient ($\mathrm{MCC}$) & $$\mathrm{MCC} = \frac{(\mathrm{TP} \cdot \mathrm{TN}) - \mathrm{FP} \cdot \mathrm{FN}}{\sqrt{(\mathrm{TP} + \mathrm{FP})(\mathrm{TP} + \mathrm{FN})(\mathrm{TN} + \mathrm{FP})(\mathrm{TN} + \mathrm{FN})}}$$ \\
		Cohen's Kappa ($\kappa$)                          & For $P_o$ the observed accuracy and $P_e$ the expected accuracy: $$\kappa = \frac{P_o - P_e}{1 - P_e}$$                                                                                                        \\                                                                                                     \\
		\bottomrule
	\end{tabularx}
	\caption{Evaluation metrics for object detection: formulae.}
	\label{tab:evaluation_metrics_formula}
\end{table}

\begin{multicols*}{2}
	\subsection{Object Detection Models for Real-Time Inference}

	\subsection{Video Analysis, Temporal Modeling and Object Tracking}

	\subsection{Real-Time Processing on Edge Devices}

	\subsection{Existing Systems}

	This section explores existing research and reviews in the field of unattended,
	or suspicious, object detection using computer vision and deep learning
	techniques. This includes a broader view on machine learning techniques for
	suspicious object detection \citep{dubey2024critical}, as well as a specific,
	novel approach to abandoned object detection using deep learning methods
	\citep{qasim2024abandoned}. The main question of the review is to determining
	how the specific implementation choices tackle the general challenges posed.

	\subsubsection[A Critical Study on Suspicious Object Detection with Images and Videos Using Machine Learning Techniques]
	{\enquote{A Critical Study on Suspicious Object Detection with Images and Videos Using Machine Learning Techniques} \protect\citep{dubey2024critical}}

	Authors \citeauthor{dubey2024critical} provide a broad study of the field,
	reviewing machine learning techniques for detecting suspicious objects. The
	authors cover a range of detection methods, from static images, real-time
	videos, and via IoT systems.

	A pertinent point raised in this review are the challenges that researchers
	faced in this domain \enquote{illumination changes, occlusion, noise, poor
		resolution, and real-time processing complexities}.

	The review compares multiple deep learning models with the most promising
	results coming from Faster-RCNN, Mask-RCNN, and variants of YOLO. Despite the
	promise of these models, the authors conclude that many limitations are still
	faced, particularly in terms of accuracy in busy scenes.

	\subsubsection[Abandoned Object Detection and Classification Using Deep Embedded Vision]
	{\enquote{Abandoned Object Detection and Classification Using Deep Embedded Vision} \protect\citep{qasim2024abandoned}}

	The article by \citeauthor{qasim2024abandoned} present a novel approach for
	identifying abandoned objects in surveillance footage. It features a two-stage
	approach, the first stage employing a ConvLSTM (Convolutional Long Short-Term
	Memory) model which combines the abilities of a CNN and LSTM to capture both
	spatial and temporal features from video data, classifying scenes as
	\enquote{suspicious} or \enquote{non-suspicious}; the second stage, given a
	suspicious scene, utilises a YOLOv8l model to classify the detected objects.

	\subsection{Model Optimisation Techniques}

	\subsection{Challenges in Surveillance Systems}

	\subsection{Identifying the Research Gap}

	The two studies provide an answer to the review's main question. The review by
	\citeauthor{dubey2024critical} establishes the persistent challenges of the
	domain: environmental factors, and, most critically for this project, the
	difficulty of \enquote{real-time processing complexities}
	\citep{dubey2024critical}. The implementation choices of researchers are a
	response to these challenges.

	A state-of-the-art example is presented by \citeauthor{qasim2024abandoned}, who
	tackle these challenges with a novel, specific implementation: a two-stage
	model \citep{qasim2024abandoned}. Their ConvLSTM classifier acts as an
	efficient first-pass filter, analysing spatio-temporal data to identify
	\enquote{suspicious} scenes. This implementation choice addresses the challenge
	of real-time processing by preventing the system from running the more
	computationally expensive YOLOv8l model on every single frame. This solution
	allows the system to achieve high accuracy (99.70\% for the localiser) by
	focusing resources only when necessary.

	However, this is where the research gap, and the justification for this
	project, becomes clear. While the \citeauthor{qasim2024abandoned} paper is
	titled \enquote{Using Deep Embedded Vision}, their experimental setup was
	conducted on a high-performance NVIDIA GeForce RTX 3090 GPU. This hardware,
	while having been superseded, remains a high-power, high-cost component not
	suitable for a typical, low-cost embedded application, revealing a gap between
	academic demonstration and practical, real-world deployment.

	Therefore, this project is justified as a critical investigation into the
	practical implementation and optimisation of an object detection model. It will
	tackle the \enquote{real-time processing complexities} identified by
	\citeauthor{dubey2024critical} by moving from a theoretical, high-performance
	environment to a constrained one.

	\section{Requirements Analysis}

	\section{System Design}

	\section{Implementation}

	\section{Testing and Evaluation}

	\section{Conclusion}
\end{multicols*}

\bibliography{report}

\end{document}
